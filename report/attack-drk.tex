The DrK is an attack against KASLR which derandomizes the kernel address space.
It has been proposed in the paper ... by ...
Most of the code necessary to perform the attack can be found on GitHub (TODO link).

The attack builds upon Intel TSX (TODO reference) and performs a timing attack to infer the state of a page (mapped vs. unmapped and executable vs. non-executable).
This works because with TSX the kernel does not get notified of Page Faults and access violations.
Furthermore the time it takes the MMU (Memory Managment Unit) to throw a Page Fault exception slightly differs from the time it takes to throw an Access Violation exception.
Using this the status of a page can be inferred.

\subsection{Intel TSX}

Intel TSX stands for the Intel Transaction Synchronization Extension(s).
It is an interface using which code can be run transactionally, i.e. segmented into clearly defined transactions, aborted and rolled back including all memory writes if so needed.

Aborting and thus a rollback happens either manually using a directive to abort the transaction or automatically in the case of an exception such as a Page Fault.

\lstset{language=C, basicstyle=\ttfamily,
        string=[b]', showspaces=false, showtabs=false,
        caption={Example code for Intel TSX}, captionpos=b}
\begin{lstlisting}
// start transaction, if rolled back
// execution continues from here
// but with a different return value
// (status code of transaction)
if (_xbegin() == _XBEGIN_STARTED) {

  ... // do work

  // commit transaction
  _xend();

  // or abort with status code
  // (=> rollback)
  _xabort(status)
} else {
  // abort handler
}
\end{lstlisting}
